
\documentclass[11pt]{article}
\usepackage{amsmath}
\usepackage{geometry}
\geometry{margin=1in}

\title{Key Formulas in Engineering Costs}
\author{}
\date{}

\begin{document}

\maketitle

\section*{1. Cost Behavior}

\begin{align}
  \text{Fixed Cost (FC)} &:\ \text{Costs that do not change with output.} \nonumber \\
  \text{Variable Cost (VC)} &:\ \text{Costs that vary proportionally with output.} \nonumber \\
  \text{Total Cost (TC)} &= FC + VC \label{eq:tc} \\
  \text{Average Cost (AC)} &= \frac{TC}{x} \label{eq:ac} \\
  \text{Marginal Cost (MC)} &= \frac{d\,VC}{d\,x} \label{eq:mc}
\end{align}

\noindent where \( x \) is the output (e.g., number of units produced).

\section*{2. Break-Even Analysis}

Suppose revenue and cost functions are:
\[
  R(x) = \rho \, x, 
  \qquad
  TC(x) = a_0 + b_0 \, x,
\]
where
\begin{itemize}
  \item \(\rho\) = unit selling price (\$/unit),
  \item \(a_0\) = total fixed cost (\$),
  \item \(b_0\) = variable cost per unit (\$/unit).
\end{itemize}

\noindent Profit (or net benefit) is:
\begin{equation}
  \Omega(x) \;=\; R(x) \;-\; TC(x) \;=\; \rho\,x \;-\; \bigl(a_0 + b_0\,x\bigr).
\end{equation}

\noindent Break-even output \( x^* \) occurs when \( \Omega(x^*) = 0 \):
\begin{equation}
  0 = \rho\,x^* \;-\; \bigl(a_0 + b_0\,x^*\bigr)
  \quad\Longrightarrow\quad
  x^* 
  = \frac{a_0}{\,\rho - b_0\,}.
  \label{eq:breakeven}
\end{equation}

\section*{3. Cost Indexes}

To update a historical cost \( C_0 \) from year 0 to year 1 using an index:
\begin{equation}
  C_1 \;=\; C_0 \; \times \; \frac{\text{Index}_1}{\text{Index}_0}.
  \label{eq:cost-index}
\end{equation}

\noindent Examples:
\begin{itemize}
  \item Operating \& Maintenance Cost Index (CPI\textsubscript{O\&M})
  \item Plant Cost Composite Index (PCCI)
\end{itemize}

\section*{4. Power-Sizing Model (Economies of Scale)}

If two pieces of equipment have known sizes and costs at the same time,
\[
  \frac{C_A}{C_B} 
  = \left(\frac{S_A}{S_B}\right)^{\,y},
\]
where
\begin{itemize}
  \item \(C_A, C_B\) = capital costs of equipment A and B,
  \item \(S_A, S_B\) = capacities (sizes) of equipment A and B,
  \item \(y\) = size-exponent (typically \(0.6 \le y \le 1.0\)).  
    \begin{itemize}
      \item If \(y < 1\), there are \emph{economies of scale}.
    \end{itemize}
\end{itemize}
Thus,
\[
  C_A 
  = C_B \,\bigl(S_A / S_B\bigr)^{\,y}.
\]

\section*{5. Scaling and Inflation Example for a Coal Plant}

\paragraph{Reference Data (Year 2010):}
\begin{itemize}
  \item Base plant (Project–B): 
    \begin{itemize}
      \item Size: \(400~\text{MW}\) 
      \item Capital cost: \$3{,}636 \;/\text{kW} \(\Rightarrow\) \$1{,}454.4~\text{M} \)
      \item Fixed O\&M: \$16.84~\text{M/year}
      \item Variable O\&M: \$4.60~\text{\$/MWh}
      \item PCCI\(_{2010}\)=100, PCCI\(_{2020}\)=172
      \item CPI\(_{\text{O\&M},2010}\)=1.9, CPI\(_{\text{O\&M},2020}\)=2.2
    \end{itemize}
\end{itemize}

\subsection*{5.1. Step 1: Scale from 400\,MW to 500\,MW at 2010 Prices}

\begin{align}
  \text{Capital cost (2010)}&: \nonumber\\
  C_{\mathrm{cap},A}^{2010}
  &= C_{\mathrm{cap},B}^{2010}
    \left(\frac{500}{400}\right)^{1.0}
  = 1{,}454.4~\text{M} \times \frac{500}{400}
  = 1{,}818~\text{M}.
  \label{eq:scale-cap} \\[1em]
  \text{Fixed O\&M (2010)}&: \nonumber\\
  \mathrm{FOM}_{A}^{2010}
  &= \mathrm{FOM}_{B}^{2010}
    \left(\frac{500}{400}\right)^{0.75}
  \approx 16.84~\text{M} \times \left(\frac{500}{400}\right)^{0.75}
  \approx 19.908~\text{M/year}.
  \label{eq:scale-fom} \\[1em]
  \text{Variable O\&M (2010)}&:
  \quad u_{\mathrm{VOM},A}^{2010} = u_{\mathrm{VOM},B}^{2010} = 4.60~\$/\text{MWh}.
  \label{eq:scale-vom}
\end{align}

\subsection*{5.2. Step 2: Inflate to 2020 Values}

\begin{align}
  \text{Capital cost (2020)}&: \nonumber\\
  C_{\mathrm{cap},A}^{2020}
  &= C_{\mathrm{cap},A}^{2010}
    \times \frac{\text{PCCI}_{2020}}{\text{PCCI}_{2010}}
  = 1{,}818~\text{M} \times \frac{172}{100}
  = 3{,}126.96~\text{M}.
  \label{eq:inf-cap} \\[1em]
  \text{Fixed O\&M (2020)}&: \nonumber\\
  \mathrm{FOM}_{A}^{2020}
  &= \mathrm{FOM}_{A}^{2010}
    \times \frac{\text{CPI}_{\text{O\&M},2020}}{\text{CPI}_{\text{O\&M},2010}}
  = 19.908~\text{M} \times \frac{2.2}{1.9}
  \approx 23.0512~\text{M/year}.
  \label{eq:inf-fom} \\[1em]
  \text{Variable O\&M (2020)}&: \nonumber\\
  u_{\mathrm{VOM},A}^{2020}
  &= u_{\mathrm{VOM},A}^{2010}
    \times \frac{\text{CPI}_{\text{O\&M},2020}}{\text{CPI}_{\text{O\&M},2010}}
  = 4.60 \times \frac{2.2}{1.9}
  \approx 5.3263~\$/\text{MWh}.
  \label{eq:inf-vom}
\end{align}

\subsection*{5.3. Step 3: Annual Cost and Revenue Functions (No Discounting)}

Assume
\begin{itemize}
  \item Plant life \( N = 50 \) years,
  \item Capacity factor \( = 0.70 \),
  \item Plant size \( = 500~\text{MW} \).
\end{itemize}

\paragraph{Annualized Capital Recovery:}
\[
  \text{CapRec}_{A}
  = \frac{C_{\mathrm{cap},A}^{2020}}{N}
  = \frac{3{,}126.96~\text{M}}{50}
  = 62.5392~\text{M/year}.
  \label{eq:caprec}
\]

\paragraph{Total Annual Cost Function:}
Let \( x = \) annual energy production (MWh). Then
\[
  x = 500~\text{MW} 
      \times 24~\frac{\text{h}}{\text{day}} 
      \times 365~\frac{\text{days}}{\text{year}}
      \times 0.70
  = 3.066 \times 10^{6}~\text{MWh/year}.
  \label{eq:energy}
\]
Fixed costs per year:
\[
  \text{FixedAnnualCost} 
  = \text{CapRec}_{A} + \mathrm{FOM}_{A}^{2020}
  = 62.5392 + 23.0512
  = 85.5904~\text{M/year}.
  \label{eq:fixed-annual}
\]
Variable cost per MWh:
\[
  u_{\mathrm{VOM},A}^{2020} = 5.3263~\$/\text{MWh}.
  \label{eq:var-cost}
\]
Therefore,
\begin{equation}
  TC_A(x) 
  = \Bigl(85.5904\Bigr)\;+\;\Bigl(5.3263 \times 10^{-6}\Bigr)\;x
  \quad\text{(in M\$ per year).}
  \label{eq:annual-cost-func}
\end{equation}

\paragraph{Revenue Function:}
If the selling price is \(\rho\) \$/MWh, then
\begin{equation}
  R_A(x) = \rho \, x \quad\text{(in \$ per year).}
  \label{eq:annual-rev}
\end{equation}

\paragraph{Profit Function:}
\begin{equation}
  \Omega_A(x) 
  = R_A(x) - TC_A(x)
  = \rho\,x - \bigl[\,85.5904 + (5.3263 \times 10^{-6})\,x \bigr].
  \label{eq:profit}
\end{equation}

\subsection*{Break-Even Price}

At break-even, \(\Omega_A(x) = 0\). Using \(x = 3.066 \times 10^{6}\) MWh/year,
\begin{align}
  0 
  &= \rho^* \times (3.066 \times 10^{6})
    \;-\; \Bigl[\,85.5904 + (5.3263 \times 10^{-6})\,(3.066 \times 10^{6})\Bigr] 
  \nonumber\\
  \rho^* 
  &= \frac{85.5904 + \bigl(5.3263 \times 10^{-6}\bigr)\,(3.066 \times 10^{6})}
         {\,3.066 \times 10^{6}\times 10^{-6}\,} 
  \label{eq:breakeven-price} \\
  &\approx \frac{85.5904 + 16.3283}{3.066}
  = \frac{101.9187}{3.066}
  \approx 32.26~\$/\text{MWh}.
  \nonumber
\end{align}

\section*{6. Additional Cost Concepts}

\begin{itemize}
  \item \textbf{Sunk Cost:} A past cash outlay that cannot be recovered; should be ignored in current decision making.
  \end{itemize}
\section*{7. Cost Estimation Models}

\begin{enumerate}
  \item \textbf{Per-Unit Model}: 
    \[
      \text{Cost} = \text{(Cost per Unit)} \times \text{Number of Units}.
    \]
\end{enumerate}

\end{document}
