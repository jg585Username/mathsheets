
\documentclass[11pt]{article}
\usepackage[utf8]{inputenc}
\usepackage{array,booktabs,geometry}
\setmainfont{Comic Sans MS}
\geometry{margin=1in}
\title{Core Definitions Road‑Map for Elementary Set Theory}
\author{}
\date{}
\begin{document}
\maketitle

\section*{How to Use}
Keep this single‑page cheat‑sheet beside you; whenever a term appears, glance here for the formal definition and common notation.

\begin{center}
\renewcommand{\arraystretch}{1.2}
\small
\begin{tabular}{@{} p{3.1cm} p{9.4cm} @{}}
\toprule
\textbf{Category} & \textbf{Definition / Notation (relative to a universal set $U$)} \\
\midrule
\multicolumn{2}{@{}l}{\textbf{Basic vocabulary}}\\
Set, element & Unordered collection of objects.  $x\in A$ means ``$x$ is an element of $A$''.\\
Empty set $\varnothing$ & Unique set with no elements.\\
Universal set $U$ & Ambient ``everything under discussion''. Needed for complements.\\
Subset / proper subset & $A\subseteq B \iff (\forall x)(x\in A\Rightarrow x\in B)$; \quad $A\subset B$ adds $A\neq B$.\\
Set equality & $A=B \iff A\subseteq B$ and $B\subseteq A$.\\
\midrule
\multicolumn{2}{@{}l}{\textbf{Set operations}}\\
Union & $A\cup B=\{x\mid x\in A\lor x\in B\}$.\\
Intersection & $A\cap B=\{x\mid x\in A\land x\in B\}$.\\
Difference & $A\setminus B=\{x\mid x\in A\land x\notin B\}$.\\
Complement & $\overline{A}=U\setminus A$.\\
Symmetric difference & $A\triangle B=(A\setminus B)\cup(B\setminus A)$.\\
\midrule
\multicolumn{2}{@{}l}{\textbf{Construction gadgets}}\\
Power set & $\mathcal P(A)=\{X\mid X\subseteq A\}$.\\
Cartesian product & $A\times B=\{(a,b)\mid a\in A,\;b\in B\}$. Extend to $A^n$.\\
Ordered pair & $(a,b)=\{\{a\},\{a,b\}\}$ (Kuratowski)—order matters.\\
\midrule
\multicolumn{2}{@{}l}{\textbf{Counting \& size}}\\
Finite/infinite & Finite $\iff$ bijection with $\{1,\ldots,n\}$ for some $n\in\mathbb N$.\\
Cardinality & $|A|$ denotes ``number of elements''.\\
Countable, uncountable & Countable = finite or bijective with $\mathbb N$; otherwise uncountable.\\
\midrule
\multicolumn{2}{@{}l}{\textbf{Functions \& relations}}\\
Binary relation & $R\subseteq A\times B$. Write $aRb$ for $(a,b)\in R$.\\
Function & Relation $f\subseteq A\times B$ with uniqueness: each $a\in A$ appears once.\\
Domain/codomain & $\mathrm{dom}(f)=A$, codomain $=B$, range $=f(A)$.\\
Injection/surjection/bijection & One‑to‑one / onto / both.\\
\midrule
\multicolumn{2}{@{}l}{\textbf{Higher‑order structures}}\\
Equivalence relation & Reflexive, symmetric, transitive; induces partitions.\\
Partition & Collection of non‑empty, pairwise‑disjoint subsets whose union is $A$.\\
Partial order & Reflexive, antisymmetric, transitive relation.\\
\midrule
\multicolumn{2}{@{}l}{\textbf{Logical laws on sets}}\\
De Morgan & $\overline{A\cup B}=\overline{A}\cap\overline{B}$ and vice‑versa.\\
Absorption & $A\cup(A\cap B)=A$.  $A\cap(A\cup B)=A$.\\
Distributive & $A\cup(B\cap C)=(A\cup B)\cap(A\cup C)$ and dual.\\
Complement laws & $A\cup\overline{A}=U$, $A\cap\overline{A}=\varnothing$, $\overline{\overline{A}}=A$.\\
\midrule
\multicolumn{2}{@{}l}{\textbf{Choice \& infinity (advanced)}}\\
Axiom of Choice (AC) & Every family of non‑empty sets has a choice function. Equivalent to Zorn's Lemma, Well‑Ordering Principle.\\
\midrule
\multicolumn{2}{@{}l}{\textbf{Notation essentials}}\\
Set‑builder & $\{x\in S\mid P(x)\}$ means ``all $x$ in $S$ satisfying $P(x)$''.\\
Logical symbols & $\forall,\exists,\neg,\land,\lor,\Rightarrow$ used inside predicates.\\
\bottomrule
\end{tabular}
\end{center}

\end{document}
